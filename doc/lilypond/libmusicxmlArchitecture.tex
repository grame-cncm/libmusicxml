
% libmusicxml2 architecture overview
% Grame, 2017

\documentclass[border=20pt]{standalone}

\usepackage{tikz}

\usetikzlibrary{math}
\usetikzlibrary{arrows.meta}

\begin{document}

\begin{tikzpicture}[scale=5]

% elements positions
% ------------------------------------------------

\tikzmath{
  \MSRAbs = 0;
  \MSROrd = 0;
  %
  \MXMLAngle = 90;
  \MXMLTreeAbs = cos(\MXMLAngle);
  \MXMLTreeOrd = sin(\MXMLAngle);
  \MusicXMLAbs = \MXMLTreeAbs * 2;
  \MusicXMLOrd = \MXMLTreeOrd * 2;
  %
  \GuidoAngle = 55;
  \GuidoAbs = cos(\GuidoAngle) * 2;
  \GuidoOrd = sin(\GuidoAngle) * 2;
  %
  \MDSRAngle = 200;
  \MDSRAbs = cos(\MDSRAngle);
  \MDSROrd = sin(\MDSRAngle);
  \MIDIAbs = \MDSRAbs * 2;
  \MIDIOrd = \MDSROrd * 2;
  %
  \LPSRAngle = 340;
  \LPSRAbs      = cos(\LPSRAngle);
  \LPSROrd      = sin(\LPSRAngle);
  \LilyPondAbs = \LPSRAbs * 2;
  \LilyPondOrd = \LPSROrd * 2;
  %
  \RandomMusicStartAbs = -1.3;
  \RandomMusicStartOrd = \MXMLTreeOrd - 0.15;
  \RandomMusicInterAbs = \MXMLTreeAbs - 0.15;
  \RandomMusicInterOrd = \MXMLTreeOrd - 0.15;
  \RandomMusicEndAbs   = \MusicXMLAbs - 0.15;
  \RandomMusicEndOrd   = \MusicXMLOrd - 0.15;
  %
  \textOutputAngle = 160;
  \textOutputRadius = 1.4;
  \textOutputAbs = cos(\textOutputAngle) * \textOutputRadius;
  \textOutputOrd = sin(\textOutputAngle) * \textOutputRadius;
  %
  \toolsStartAbs = \MusicXMLAbs - 0.05;
  \toolsStartOrd = \MusicXMLOrd - 0.15;
  \toolsInterAbs = \MXMLTreeAbs - 0.05;
  \toolsInterOrd = \MXMLTreeOrd - 0.1;
  \toolsEndAbs   = \textOutputAbs + 0.15;
  \toolsEndOrd   = \textOutputOrd + 0.0;
  %
  \xmlToGuidoInterangle = (\MXMLAngle + \GuidoAngle) / 2;
  \xmlToGuidoInterRadius = 0.6;
  \xmlToGuidoStartAbs = \MusicXMLAbs + 0.15;
  \xmlToGuidoStartOrd = \MusicXMLOrd - 0.15;
  \xmlToGuidoInterAbs = cos(\xmlToGuidoInterangle) * \xmlToGuidoInterRadius;
  \xmlToGuidoInterOrd = sin(\xmlToGuidoInterangle) * \xmlToGuidoInterRadius;
  \xmlToGuidoEndAbs   = \GuidoAbs - 0.15;
  \xmlToGuidoEndOrd   = \GuidoOrd - 0.15;
  %
  \xmlToLilyPondInterIAngle  = mod((\MXMLAngle + \LPSRAngle) * 2 / 3, 360);
  \xmlToLilyPondInterIIAngle = mod((\MXMLAngle + \LPSRAngle) * 1 / 3, 360);
  \xmlToLilyPondInterRadius = 0.2;
  \xmlToLilyPondStartAbs = \MusicXMLAbs + 0.05;
  \xmlToLilyPondStartOrd = \MusicXMLOrd - 0.15;
  \xmlToLilyPondInterIAbs = cos(\xmlToLilyPondInterIAngle) * \xmlToLilyPondInterRadius;
  \xmlToLilyPondInterIOrd = sin(\xmlToLilyPondInterIAngle) * \xmlToLilyPondInterRadius;
  \xmlToLilyPondInterIIAbs = cos(\xmlToLilyPondInterIIAngle) * \xmlToLilyPondInterRadius;
  \xmlToLilyPondInterIIOrd = sin(\xmlToLilyPondInterIIAngle) * \xmlToLilyPondInterRadius;
  \xmlToLilyPondEndAbs   = \LilyPondAbs - 0.15;
  \xmlToLilyPondEndOrd   = \LilyPondOrd + 0.15;
  %
  \titleAbs = -2.2;
  \titleOrd = 2.0;
  %
  \tableOrd = -1.7;
  %
  \bottomLineOrd = -2.4;
  %
  \legendAbs = \titleAbs;
  \legendOrd = -1.4;
  %
  \dateAbs = 1.3;
}


% Draw the title  
% ------------------------------------------------

\node[right,scale=2] (title) at (\titleAbs,\titleOrd) {libmusicxml2:};

\node[right,scale=2] (title) at (\titleAbs,\titleOrd-0.15) {architecture overview};

\node[right,scale=1] at (\titleAbs,\titleOrd-0.30) {~(light gray indicates items not yet available)};


% Draw the elements
% ------------------------------------------------

\node[align=center,style={circle,minimum size=80,fill=green!20}]
  (MSR) at (\MSRAbs,\MSROrd) {MSR\\(graph)};

\node[align=center,style={rectangle,rounded corners=10pt,minimum size=80,fill=red!15}]
  (MusicXML)
  at (\MusicXMLAbs,\MusicXMLOrd) {MusicXML\\(text)};
\node[align=center,style={circle,minimum size=80,fill=red!15}]
  (MXMLTree)
  at (\MXMLTreeAbs,\MXMLTreeOrd) {xmlelement\\tree};
  
\node[align=center,style={rectangle,rounded corners=10pt,minimum size=80,fill=brown!20}]
  (Guido) at (\GuidoAbs,\GuidoOrd) {Guido\\(text)};
  
\node[align=center,style={rectangle,rounded corners=10pt,minimum size=80,fill=orange!20}]
  (LilyPond)
  at (\LilyPondAbs, \LilyPondOrd) {LilyPond\\(text)};
\node[align=center,style={circle,minimum size=80,fill=orange!20}]
  (LPSR)
  at (\LPSRAbs, \LPSROrd) {LPSR\\(graph)};


\node[align=center,style={rectangle,rounded corners=10pt,minimum size=80,fill=blue!15}]
  (MIDI)
  at (\MIDIAbs,\MIDIOrd) {MIDI\\(binary)};
\node[align=center,style={circle,minimum size=80,fill=blue!15,text=gray!80}]
  (MDSR)
  at (\MDSRAbs,\MDSROrd) {MDSR\\(?)};

\node[align=center]
  (textOutput)
  at (\textOutputAbs,\textOutputOrd) {\includegraphics[scale=0.15]{TextOutputIcon.png}};


% Draw the arrows between the circles
% ------------------------------------------------

\draw [<->,thick,{Stealth[length=10pt]}-{Stealth[length=10pt]}] (MusicXML) -- (MXMLTree);
\draw [->,thick,-{Stealth[length=10pt]}] (MXMLTree) -- (MSR);

\draw [->,thick,-{Stealth[length=10pt]}] (MSR) -- (LPSR);
\draw [->,thick,-{Stealth[length=10pt]}] (LPSR) -- (LilyPond);

\draw [->,very thick,-{Stealth[length=10pt]},loosely dashed,color=gray!50] (MIDI) -- (MDSR);
\draw [->,very thick,-{Stealth[length=10pt]},loosely dashed,color=gray!50] (MDSR) -- (MSR);


% Draw the tools
% ------------------------------------------------

\filldraw [gray!30] 
  (\toolsStartAbs, \toolsStartOrd) circle [radius=0pt]
  (\toolsInterAbs,\toolsInterOrd) circle [radius=0pt] 
  (\toolsEndAbs,\toolsEndOrd) circle [radius=0pt];
  
\draw [->,thick,red,-{Stealth[length=10pt]},anchor=center]
  (\toolsStartAbs, \toolsStartOrd)
  ..
  node[above,pos=0.85,sloped] {tools}
  controls (\toolsInterAbs,\toolsInterOrd)
  .. 
  (\toolsEndAbs,\toolsEndOrd);


% Draw RandomMusic
% ------------------------------------------------

\filldraw [gray!30] 
  (\RandomMusicStartAbs, \RandomMusicStartOrd) circle [radius=1pt]
  (\RandomMusicInterAbs,\RandomMusicInterOrd) circle [radius=0pt] 
  (\RandomMusicEndAbs,\RandomMusicEndOrd) circle [radius=0pt];
  
\draw [->,thick,red,-{Stealth[length=10pt]},anchor=center]
  (\RandomMusicStartAbs, \RandomMusicStartOrd)
  ..
  node[above,pos=0.175,sloped] {RandomMusic}
  controls (\RandomMusicInterAbs,\RandomMusicInterOrd)
  .. 
  (\RandomMusicEndAbs,\RandomMusicEndOrd);


% Draw xml2guido
% ------------------------------------------------

\filldraw [gray!30] 
  (\xmlToGuidoStartAbs, \xmlToGuidoStartOrd) circle [radius=0pt]
  (\xmlToGuidoInterAbs,\xmlToGuidoInterOrd) circle [radius=0pt] 
  (\xmlToGuidoEndAbs,\xmlToGuidoEndOrd) circle [radius=0pt];
  
\draw [->,thick,red,-{Stealth[length=10pt]},anchor=center]
  (\xmlToGuidoStartAbs, \xmlToGuidoStartOrd)
  ..
  node[above,pos=0.815,sloped] {xmlToGuido}
  controls (\xmlToGuidoInterAbs,\xmlToGuidoInterOrd)
  .. 
  (\xmlToGuidoEndAbs,\xmlToGuidoEndOrd);


% Draw xml2lilypond  
% ------------------------------------------------

\filldraw [gray!30] 
  (\xmlToLilyPondStartAbs,\xmlToLilyPondStartOrd) circle [radius=0pt]
  (\xmlToLilyPondInterIAbs,\xmlToLilyPondInterIOrd) circle [radius=0pt] 
  (\xmlToLilyPondInterIIAbs,\xmlToLilyPondInterIIOrd) circle [radius=0pt] 
  (\xmlToLilyPondEndAbs,\xmlToLilyPondEndOrd) circle [radius=0pt];
  
\draw [->,thick,red,-{Stealth[length=10pt]},anchor=center]
  (\xmlToLilyPondStartAbs, \xmlToLilyPondStartOrd)
  ..
  node[above,pos=0.685,sloped] {xml2lilypond}
  controls 
    (\xmlToLilyPondInterIAbs,\xmlToLilyPondInterIOrd)
    and
    (\xmlToLilyPondInterIIAbs,\xmlToLilyPondInterIIOrd)
  .. 
  (\xmlToLilyPondEndAbs,\xmlToLilyPondEndOrd);
  
  
% Draw the table
% ------------------------------------------------

\node[right,scale=1] (table) at (\legendAbs,\tableOrd) {
\def \contentsWidth{1.25\textwidth}
\def \arraystretch{1.3}
%
\begin{tabular}[t]{lp{\contentsWidth}}
{Entity} & {Description} \tabularnewline[0.5ex]
\hline\\[-3.0ex]
%
xmlelement tree & a tree representing the MusicXML markups such as {\tt <part-list>}, {\tt <time>} and {\tt <note>}
\tabularnewline

MSR & Music Score Representation, in terms of part groups, parts, staves, voices, notes, \ldots
\tabularnewline

LPSR & LilyPond Score Representation, i.e. MSR plus LilyPond-specific items such as {\tt $\backslash$score} blocks
\tabularnewline

MDSR & MIDI Score Representation, to be designed
\tabularnewline

RandomMusic & generates an xmlelement tree containing random music and writes it as MusicXML
\tabularnewline

tools & a set of other demo programs such as {\tt countnotes} and {\tt partsummary}, showing the use of the library
\tabularnewline

xml2lilypond & performs the 4 hops from MusicXML to LilyPond to translate the former into the latter
\tabularnewline

\end{tabular}
};


% Draw the note
% ------------------------------------------------

\node[right,scale=1] (note) at (\legendAbs,\bottomLineOrd+0.0) {$\bullet$ Note: LilyPond can generate MIDI files from its input};


% Draw the date
% ------------------------------------------------

\node[right,scale=1] (date) at (\dateAbs,\bottomLineOrd) {v2.2/0.1.0, February 2017};


\end{tikzpicture}
  
\end{document}
