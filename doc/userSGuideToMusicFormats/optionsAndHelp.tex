% -------------------------------------------------------------------------
\subsection{OAH basics}
% -------------------------------------------------------------------------

\begin{itemize}
\item OAH (Options And Help) is supposed to be pronounced something close to "whaaaah!"
    The intonation is left to the speaker, though...
    And as the saying goes: "OAH? oahy not!"

\item options handling is organized as a hierarchical, instrospective set of classes.
    An options and its corresponding help are grouped in a single object.

\item the options can be supplied thru:
  \begin{itemize}
    \item the command line, in argv.
        This allows for mixed options and arguments in any order, à la GNU;
    \item the API function such as musicxmlfile2lilypond(), in an options vector.
  \end{itemize}

\item oahElement is the super-class of all options types, including groups and subgroups.
    It contains a short name and a long name, as well as a decription.
    Short and long names can be used and mixed at will in the command line
    and in option vectors (API),
    as well as '-' and '--'.
    The short name is mandatory, but the long name may be empty
    if the short name is explicit enough.

\item prefixes such '-t=' and -help=' allow for a contracted form of options.
    For example, -t=meas,notes is short for '-t-meas, -tnotes'.
    A oahPrefix contains the prefix name, the ersatz by which to replace it,
    and a description.

\item a oahHandler contains a list of oahGroup's, each handled
    in a pair or .h/.cpp files such as msrOah.h and msrOahGroup.cpp,
    and a list of options prefixes.

\item a oahGroup contains a list of oahSubGroup's
    and an upLink to the containing oahHandler.

\item a oahSubGroup contains a list of oahAtom's
    and an upLink to the containing oahGroup.

\item each oahAtom contains an atomic option and the corresponding help,
    and an upLink to the containing oahSubGroup.

\end{itemize}


